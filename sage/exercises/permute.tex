%%%%(c)
%%%%(c)  This file is a portion of the source for the textbook
%%%%(c)
%%%%(c)    Abstract Algebra: Theory and Applications
%%%%(c)    by Thomas W. Judson
%%%%(c)
%%%%(c)    Sage Material
%%%%(c)    Copyright 2011 by Robert A. Beezer
%%%%(c)
%%%%(c)  See the file COPYING.txt for copying conditions
%%%%(c)
%%%%(c)
These exercises are designed to help you become familiar with permutation groups in Sage.
%
\begin{sageverbatim}\end{sageverbatim}
%
\sageexercise{1}%
Create the full symmetric group $S_{10}$ with the command \verb?G = SymmetricGroup(10)?.
\begin{sageverbatim}\end{sageverbatim}
%
\sageexercise{2}
Create elements of \verb?G? with the following (varying) syntax.  Pay attention to commas, quotes, brackets, parentheses.  The first two use a string (characters) as input, mimicking the way we write permuations (but with commas).  The second two use a list of tuples.\par\noindent
(a) \verb?a = G("(5,7,2,9,3,1,8)")?\\
(b) \verb?b = G("(1,3)(4,5)")?\\
(c) \verb?c = G([(1,2),(3,4)])?\\
(d) \verb?d = G([(1,3),(2,5,8),(4,6,7,9,10)])?
\begin{sageverbatim}\end{sageverbatim}
%
\sageexercise{3}
%
Compute $a^3$, $bc$, $ad^{-1}b$.
\begin{sageverbatim}\end{sageverbatim}
%
\sageexercise{4}
%
Compute the orders of each of these four individual elements (\verb?a? through \verb?d?) using a single permutation group element method.
\begin{sageverbatim}\end{sageverbatim}
%
\sageexercise{5}
Use the permutation group element method \verb?.sign()? to determine if $a,b,c,d$ are even or odd permutations.
\begin{sageverbatim}\end{sageverbatim}
%
\sageexercise{6}
Create two cyclic subgroups of $G$ with the commands:
%
\begin{itemize}
\item\verb?H = G.subgroup([a])?
\item\verb?K = G.subgroup([d])?
\end{itemize}
%
List, and study, the elements of each subgroup.  List the size and number of all of the subgroups of $K$ (without using Sage), and construct a subgroup of $K$ of size 10 using Sage.
\begin{sageverbatim}\end{sageverbatim}
%
\sageexercise{7}
More complicated subgroups can be formed by using two or more generators.  Construct a subgroup $L$ of $G$ with the command \verb?L = G.subgroup([b,c])?.  Compute the order of $L$ and list all of the elements of $L$.
\begin{sageverbatim}\end{sageverbatim}
%
\sageexercise{8}
Construct the group of symmetries of the tetrahedron (also the alternating group on 4 symbols, $A_4$) with the command \verb?A=AlternatingGroup(4)?.  Using tools such as orders of elements, and generators of subgroups, see if you can find \emph{all of} the subgroups of $A_4$ (each one exactly once).  Provide a nice summary as your answer - not just piles of output.  So use Sage as a tool, as needed, but basically your answer will be a concise paragraph and/or table.  This is the one part of this assignment without clear, precise directions, so spend some time on this portion to get it right.  Hint: no subgroup of $A_4$ requires more than two generators.
\begin{sageverbatim}\end{sageverbatim}
%
\sageexercise{9}
Save your work, and then see if you can crash your Sage session with the commands:
%
\begin{itemize}
\item\verb?N = G.subgroup([b,d])?
\item\verb?N.list()?
\end{itemize}
%
How big is $N$?
\begin{sageverbatim}\end{sageverbatim}
%
\sageexercise{10}
Answer the five questions above about the permutations of the cube expressed as permutations of the 8 vertices.
\begin{sageverbatim}\end{sageverbatim}
%
