%%%%(c)
%%%%(c)  This file is a portion of the source for the textbook
%%%%(c)
%%%%(c)    Abstract Algebra: Theory and Applications
%%%%(c)    by Thomas W. Judson
%%%%(c)
%%%%(c)    Sage Material
%%%%(c)    Copyright 2011 by Robert A. Beezer
%%%%(c)
%%%%(c)  See the file COPYING.txt for copying conditions
%%%%(c)
%%%%(c)
\begin{sageverbatim}\end{sageverbatim}
%
\sageexercise{1}%
In the analysis of Example~\extref{example:galois:galois_x4-2}{7}{quartic subgroup lattice example} with Sage, two subgroups of order 2 and one subgroup of order 4 were not analyzed.  Determine the fixed fields of these three subgroups.
\begin{sageverbatim}\end{sageverbatim}
%
\sageexercise{2}%
Build the splitting field of $p(x)=x^3-6x^2+12x-10$ and then determine the Galois group of $p(x)$ as a concrete group of explicit permutations.  Build the lattice of subgroups of the Galois group, again using the same explicit permutations.  Using the fundamental theorem of Galois theory, construct the subfields of the splitting field.  Include your supporting documentation in your submitted Sage worksheet.  Also, submit a written component of this assignment containing a complete layout of the subgroups and subfields, written entirely with mathematical notation and with no Sage commands, designed to illustrate the correspondence between the two.  All you need here is the graphical layout, suitably labeled --- the Sage worksheet will substantiate your work.
\begin{sageverbatim}\end{sageverbatim}
%
\sageexercise{3}%
The polynomial $x^5-x-1$ has all of the symmetric group $S_5$ as its Galois group.  Because $S_5$ is not solvable, we know this polynomial to be an example of a quintic polynomial that is not solvable by radicals.  Unfortunately, asking Sage to compute this Galois group takes far too long.  So this exercise will simulate that experience with a slightly smaller example.\par
%
Consider the polynomial $p(x)=x^4+x+1$.\\
%
(a) Build the splitting field of $p(x)$ one root at a time.  Create an extension, factor there, discard linear factors, use the remaining irreducible factor to extend once more.  Repeat until $p(x)$ factors completely.  Be sure to do a final extension via just a linear factor.  This is a little silly, and Sage will seem to ignore your final generator (so you will want to setermine what it is equivalent to in terms of the previous gfenerators).  Directions below depend on taking this extra step.\\
%
(b) Factor the original polynomial over the final extension field in the tower.  What is boring about this factorization in comparison to some other examples we have done?\\
%
(c) Construct the full tower as an absolute field over ${\mathbb Q}$.  From the degree of this extension and the degree of the original polynomial, infer the Galois group of the polynomial.\\
%
(d) Using the mappings that allow you to translate between the tower and the absolute field (obtained from the \verb?.structure()? method), choose one of the roots (any one) and express it in terms of the single generator of the absolute field.  Then reverse the procedure and express the single generator of the absolute field in terms of the roots in the tower.\par
%
(e) Compute the group of automorphisms of the absolute field (but don't display the whole group in what you submit).  Take all four roots (including your silly one from the last step of the tower construction) and apply each field automorphism to the four roots (creating the guaranteed permutations of the roots).  Comment on what you see.\\
%
(f)  There is one nontrivial automorphism that has an especially simple form (it is the second one for me) when applied to the generator of the absolute field.  What does this automorphism do to the roots of $p(x)$?\\
%
(g)  Consider the extension of ${\mathbb Q}$ formed by adjoining just one of the roots.  This is a subfield of the splitting field of the polynomial, so is the fixed field of a subgroup of the Galois group.  Give a simple description of the corresponding subgroup using language we typically only apply to permutation groups.
\begin{sageverbatim}\end{sageverbatim}
%
\sageexercise{4}%
Return to the splitting field of the quintic discussed in the introduction to the previous problem ($x^5-x-1$).  Create the first two intermediate fields by adjoining two roots (one at a time).  But instead of factoring at each step to get a new irreducible polynomial, \emph{divide} by the linear factor you \emph{know} is a factor.  In general, the quotient might factor further, but in this exercise presume it does not.  In other words, act as if your quotient by the linear factor is irreducible.  If it is not, then the \texttt{NumberField} command should complain (which it won't).\par
%
After adjoining two roots, create the extension producing a third root, and do the division.  You should now have a quadratic factor.  Assuming the quadratic is irreducible (it is) argue that you have enough evidence to establish the order of the Galois group, and hence can determine \emph{exactly} which group it is.\par
%
You can try to use this quadratic factor to create one more step in the extensions, and you will arrive at the splitting field, as can be seen with logic or division.  However, this could take a long time to complete (save your work beforehand!).  You can try passing the \verb?check=False? argument to the \verb?NumberField? command --- this will bypass checking irreducibility.
\begin{sageverbatim}\end{sageverbatim}
%
\sageexercise{5}%
Create the finite field of order $3^6$, letting Sage supply the default polynomial for its construction.  The polynomial $x^6+x^2+2*x+1$ is irreducible over this finite field.  Check that this polynomial splits in the finite field, and then use the \verb?.roots()? method to collect the roots of the polynomial.  Get the group of automorphisms of the field with the \verb?End()? command.\par
%
You now have all of the pieces to associate each field automorphism with a permutation of the roots.  From this, identify the Galois group and all of its subgroups.  For each subgroup, determine the fixed field.  You might find the roots easier to work with if you use the \verb?.log()? method to identify them as powers of the field's multiplicative generator.\par
%
Your Galois group in this example will be abelian.  So every subgroup is normal, and hence any extension is also normal.  Can you give extend this example by choosing a nontrivial intermediate field with a nontrivial irreducible polynomial that has all of its roots in the intermediate field and a nontrivial irreducible polynomial with none of its roots in the intermediate field?
%
Your results here are ``typical'' in the sense that the particular field or irreducible polynomial makes little difference in the qualitative nature of the results.
\begin{sageverbatim}\end{sageverbatim}
%
\sageexercise{6}%
The splitting field for the irreducible polynomial $p(x)=x^7-7x+3$ has degree 168 (hence this is the order of the Galois group).  This polynomial is derived from an ``Elkies trinomial curve,'' a hyperelliptic curve (below) that produces polynomials with interesting Galois groups:
$$y^2 = x(81x^5 + 396x^4 + 738x^3 + 660x^2 + 269x + 48)$$\par
%
For $p(x)$ the resulting Galois group is $PSL(2,7)$, a simple group.  If $SL(2,7)$ is all $2\times 2$ matrices over ${\mathbb Z}_7$ with determinant 1, then $PSL(2,7)$ is the quotient by the subgroup $\{I_2,-I_2\}$.  It is the second-smallest non-abelian simple group (after $A_5$).\par
%
See how far you can get in using Sage to build this splitting field.  A degree 7 extension will yield one linear factor, and a subsequent degree 6 extension will yield two linear factors, leaving a quartic factor.  Here is where the computations begin to slow down.  If we believe that the splitting field has degree 168, then we know that adding a root from this degree 4 factor will get us to the splitting field.  Creating this extension may be possible computationally, but verifying that the quartic splits into linear factors here seems to be infeasible.
\begin{sageverbatim}\end{sageverbatim}
%
\sageexercise{7}%
Return to ~\extref{example:galois:galois_x4-2}{7}{quartic subgroup lattice example}, and the complete list of subfields obtainable from the \verb?.subfields()? method applied to the flattened tower.  As mentioned, these are technically not subfields, but do have embeddings into the tower.  Given two subfields, their respective primitive elements are embedded into the tower, with an image that is a linear combination of powers of the primitive element for the tower.\par
%
If one subfield is contained in the other, then the image of the primitive element for the smaller field should be a linear combination of the (appropriate) powers of the image of the primitive element for the larger field.  This is a linear algebra computation that should be possible in the tower, relative to the power basis for the whole tower.\par
%
Write a procedure to determine if two subfields are related by one being a subset of the other.  Then use this procedure to create the lattice of subfields.  The eventual goal would be a graphical display of the lattice, using the existing plotting facilities available for lattices, similar to the top half of Figure~\extref{Galois3}{23.3}{subfield diagram}.  This is a ``challenging'' exercise, which is code for ``it has not been tested.''
\begin{sageverbatim}\end{sageverbatim}
%

