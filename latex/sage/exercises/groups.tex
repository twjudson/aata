%%%%(c)
%%%%(c)  This file is a portion of the source for the textbook
%%%%(c)
%%%%(c)    Abstract Algebra: Theory and Applications
%%%%(c)    by Thomas W. Judson
%%%%(c)
%%%%(c)    Sage Material
%%%%(c)    Copyright 2011 by Robert A. Beezer
%%%%(c)
%%%%(c)  See the file COPYING.txt for copying conditions
%%%%(c)
%%%%(c)
These exercises are about becoming comfortable working with groups in Sage.
\begin{sageverbatim}\end{sageverbatim}
%
\sageexercise{1}%
Create the groups \verb?CyclicPermutationGroup(8)? and \verb?DihedralGroup(4)? and give the two groups names of your choosing.  We will understand these constructions better shortly, but for now just understand that they are both groups.
\begin{sageverbatim}\end{sageverbatim}
%
\sageexercise{2}%
Check that the groups have the same size with the \verb?.order()? method.  Determine which is abelian, and which is not, by using the \verb?.is_abelian()? method.
\begin{sageverbatim}\end{sageverbatim}
%
\sageexercise{3}%
Use the \verb?.cayley_table()? method to create the Cayley table for each group.
\begin{sageverbatim}\end{sageverbatim}
%
\sageexercise{4}%
Write a nicely formatted discussion (Shift-click on a blue bar to bring up the mini-word-processor, use dollar signs to embed bits of \TeX) identifying differences between the two groups that are discernible in properties of their Cayley tables.  In other words, what is {\em different} about these two groups that you can ``see'' in the Cayley tables?
\begin{sageverbatim}\end{sageverbatim}
%
\sageexercise{5}%
For each group, use the \verb?.subgroups()? method to locate a largest subgroup that is not the entire group, and then use the \verb?.list()? method of the subgroup to get a list of elements (which you might save as \verb?elts?).\par
%
Now, \verb?.cayley_table(elements=elts)? for the original group will produce the Cayley table of the subgroup.  What properties of this table would you check to see if the output is correct?
\begin{sageverbatim}\end{sageverbatim}
%
\sageexercise{6}%
The \verb?.subgroup(elt_list)? method of the original group will create the smallest subgroup containing specified elements of the group, when given the elements as a list \verb?elt_list?.  Discover the shortest list of elements necessary to recreate the subgroup you used in the previous exercise.  The equality comparison, \verb?==?, can be used to test if two subgroups are equal.
\begin{sageverbatim}\end{sageverbatim}
%
