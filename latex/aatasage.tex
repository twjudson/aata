\documentclass[12pt,oneside]{book}

\usepackage{graphicx}  %% stop errors on config file load
\usepackage{amsmath}
\usepackage{amssymb}
\usepackage{url}

\usepackage{url}
\usepackage[pdftex]{hyperref}
\hypersetup{
    linktoc=page,
    colorlinks=true,
    linkcolor=blue,
    urlcolor=blue,
    pdftitle={Sage for Abstract Algebra},
    pdfauthor={Robert A. Beezer}
}

\usepackage{geometry}
\geometry{letterpaper,total={6in,9in}}

% We request SageTeX text output
%   but likely format it via our own hacked version of SageTeX
\usepackage{sagetex}
\renewcommand{\sageexampleincludetextoutput}{True}

% PDF of all Sage content - set discussions to true
%   Use the *.aux files from a legitimate full AATA run for the xr package
% Worsheets of just exercises - set discussions to false
%   Hand-coded references will come along
\newboolean{discussions}\setboolean{discussions}{true}        % false to get just exercises


% The xr package does not work with tex4ht translation
% so when we subset Sage exercises to a worksheet, any
% references in the exercises fail.
% So for a PDF of Sage content the xr mechanism is fine
% but for exercises only, we pick up hand-coded references
% see the note below for debugging/checking these hand-coded
% values.
%
% \extref  parameters
% #1 = label defined in master AATA
% #2 = actual resolved value done by hand
% #3 = note for debugging the #1==#2 correspondence
\ifthenelse{\boolean{discussions}}
{%
\usepackage{xr}%
\externaldocument{aata}%
\newcommand{\extref}[3]{\ref{#1}}%
}{%
\newcommand{\extref}[3]{#2}%
}

% To debug hand-coded references in the exercises,
% copy source to a temporary directory, comment-out
% above definition, and un-comment the one below.
% Copy over *.aux files from a legitimate run on whole
% book (or leave in place), then do a PDF run on this file
% and search PDF for "content note"
% 
% \usepackage{xr}
% \externaldocument{aata}
% \newcommand{\extref}[3]{
% \\
% \fbox{\fbox{\parbox{\textwidth}{
% Reference Correspondence Report\\
% Resolves as: \ref{#1}\\
% Number used: #2\\
% Content note: #3
% }}}
% }


% pull in sage material for a whole chapter
% #1 = chapter number
% #2 = chapter title
% #3 = chapter file name
\newcommand{\sageworksheet}[3]{
\setcounter{chapter}{#1}
\addtocounter{chapter}{-1}
\chapter{#2}%
\ifthenelse{\boolean{discussions}}{%
\sagesection{Discussion}%
\input{./sage/#3.tex}%
\sagesection{Exercises}%
\input{./sage/exercises/#3.tex}%
}{%
Exercises\par
From: \href{http://abstract.pugetsound.edu}{Abstract Algebra: Theory and Applications}\\
Sage exercises Copyright 2011-2013, Robert A. Beezer, \href{http://www.gnu.org/licenses/fdl.html}{GFDL} License\par
\input{./sage/exercises/#3.tex}%
}
}

% sage material comes in two parts:
% discussion and exercises
% we use sections to delineate these parts
% sage discussions has it's own internal sectioning
\setcounter{tocdepth}{0}
\newcommand{\sagesection}[1]{\section{#1}}
\newcommand{\sagesubsection}[1]{\subsection{#1}}

% sage exercises cannot be formatted as lists
% since we cannot bury top-level compute cells
% in a subsidiary list, so we do the numbering ourselves
% precede/finish each exercise with a sageverbatim environment
% Usage: \sageexercise{number}
\newcommand{\sageexercise}[1]{\noindent\textbf{#1}\ \ }


\begin{document}
%%
\thispagestyle{empty}
\begin{center}
\vspace*{\stretch{2}}
{\fontsize{24}{24}\selectfont%
Sage for Abstract Algebra\\[24pt]
A Supplement to\\[6pt]
Abstract Algebra, Theory and Applications}
\par
\vspace*{\stretch{2}}
%
{\large by\\Robert A.\ Beezer\\Department of Mathematics and Computer Science\\University of Puget Sound}\par
%
\vspace*{\stretch{1}}
%
%% Hand-edit below for POD versions
%
\today\\
Copyright Robert A. Beezer\\
GNU Free Documentation License\\[12pt]
Sage Version 5.11\\
AATA Version 2013-14
%
\vspace*{\stretch{1}}
\end{center}
\newpage
%%
\thispagestyle{empty}
\vspace*{\stretch{3}}\noindent
Copyright 2011-13 Robert A. Beezer. Permission is granted to copy, distribute
and/or modify this document under the terms of the GNU Free
Documentation License, Version 1.2 or any later version published by the
Free Software Foundation; with no Invariant Sections, no Front-Cover
Texts, and no Back-Cover Texts. A copy of the license is included in the
section entitled ``GNU Free Documentation License''.
\par
\vspace*{\stretch{1}}
\newpage
%%
\thispagestyle{empty}
\tableofcontents
\newpage
%%
\thispagestyle{empty}
%%
\chapter*{Preface}
\addcontentsline{toc}{chapter}{Preface}
%%
This supplement explains how to use the open source software Sage to aid in your understanding of abstract algebra.  Each section aims to explain Sage commands relevant to some topic in abstract algebra.\par
%
This material has been extracted from Tom Judson's open content textbook, {\sl Abstract Algebra: Theory and Applications} and is organized according to the chapters of that text.  It also makes frequent reference to examples, definitions and theorems from that text.  So this may be useful for learning Sage if you already know some abstract algebra.  However, if you are simultaneously learning abstract algebra you will find the textbook useful.\par
%
In any event, the best way to use this material is in its electronic form.  The content of the text, plus the material here about Sage are available together in an electronic form as a collection of Sage worksheets.  These may be uploaded to public Sage servers, such as \url{sagenb.org}, or used with a Sage installation on your own computer.  In this form, the examples printed here are ``live'' and may be edited and evaluated with new input.  Also, you can add your own notes to the text with the built-in word processor available in the Sage notebook.\par
%
For more information consult:
%
\begin{itemize}
%
\item {\sl Abstract Algebra: Theory and Applications}, \url{http://abstract.pugetsound.edu}
%
\item {\sl Sage}, \url{http://sagemath.org}
%
\end{itemize}
%
You will also want to be sure you are using a version of Sage that corresponds to the material here.  Sage is constantly being improved, and we regularly perform automatic testing of the examples here on the most recent releases.  If you are reading the electronic version, you can run the \verb?version()? command below to see which version of Sage you are using (click on the blue ``evaluate'' link).
%
\begin{sageverbatim}
sage: version()
'Sage Version 5.11, Release Date: 2013-08-13'
\end{sageverbatim}
%
\begin{sageverbatim}
\end{sageverbatim}
%
% Chapter 1
\sageworksheet{1}{Preliminaries}{sets}
% Chapter 2
\sageworksheet{2}{The Integers}{integers}
% Chapter 3
\sageworksheet{3}{Groups}{groups}
% Chapter 4
\sageworksheet{4}{Cyclic Groups}{cyclic}
% Chapter 5
\sageworksheet{5}{Permutation Groups}{permute}
% Chapter 6
\sageworksheet{6}{Cosets and Lagrange's Theorem}{cosets}
% Chapter 7
\sageworksheet{7}{Cryptography}{crypt}
%
% Chapter 8
% Algebraic Coding Theory
%
% Chapter 9
\sageworksheet{9}{Isomorphisms}{isomorph}
% Chapter 10
\sageworksheet{10}{Normal Subgroups and Factor Groups}{normal}
% Chapter 11
\sageworksheet{11}{Homomorphisms}{homomorph}
% Chapter 12
% Matrix Groups and Symmetry
%
% Chapter 13
\sageworksheet{13}{The Structure of Groups}{struct}
% Chapter 14
\sageworksheet{14}{Group Actions}{actions}
% Chapter 15
\sageworksheet{15}{The Sylow Theorems}{sylow}
% Chapter 16
\sageworksheet{16}{Rings}{rings}
% Chapter 17
\sageworksheet{17}{Polynomials}{poly}
% Chapter 18
\sageworksheet{18}{Integral Domains}{domains}
% Chapter 19
\sageworksheet{19}{Lattices and Boolean Algebras}{boolean}
% Chapter 20
\sageworksheet{20}{Vector Spaces}{vect}
% Chapter 21
\sageworksheet{21}{Fields}{fields}
% Chapter 22
\sageworksheet{22}{Finite Fields}{finite}
% Chapter 23
\sageworksheet{23}{Galois Theory}{galois}
%
{\small\input{./gfdl.tex}}
\end{document}